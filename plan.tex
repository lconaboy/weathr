\documentclass[12pt,preprint]{article}

\usepackage[a4paper, total={7in, 9in}]{geometry}

\newcommand{\elnino}[0]{\emph{El Ni\~no}}
\newcommand{\enso}[0]{\emph{ENSO}}
\newcommand{\workpackage}[2]{\textsc{#1} --- #2}

% TODO Better title
\title{Project Plan}
\date{}

\begin{document}

\section{Statement of aims}

\elnino{} is known to have significant local effects, and some
research claims measurable distant effects in Europe and Africa. It is
the aim of this project to investigate the causal relation -- if any ---
between \elnino{} events upon European and African weather. We will
use data from EUMETSAT, particularly that which focuses on Africa, and
apply image processing techniques in order to extract useful
information such as how cloud coverage varies over time.

\section{Work packages}

Work packages are presented in the order that they are expected to
begin; i.e. generally, any given work package can depend on all those
before it.

\vspace{.1cm}

\workpackage{Collect data}{We need both greyscale and multi-band data
  for many years.}

\vspace{.1cm}

\workpackage{Cloud free image}{To calculate the cloud coverage, we
  need to know how a pixel looks without cloud cover.}
\workpackage{Investigate algorithms}{Many algorithms exist for
  thresholding, each with benefits and negatives.}
\workpackage{Thresholding}{Threshold images using a manually selected
  threshold.} \workpackage{Otsu}{Otsu's method selects a threshold
  automatically for bimodal data.} \workpackage{Multiband}{An improved
  method uses multiple bands of any image and identifies clouds by
  comparing peaks in all bands.} \workpackage{False colour
  CFI}{Produce false colour CFI using the bands above.}
\workpackage{Seasonal CFI}{Over the year, vegetation in Africa varies
  in position. We can capture this in our CFI by producing a season
  CFI.} \workpackage{Treshold data with CFI}{With a CFI we can
  identify the cloud coverage by comparing each pixel in the input
  with the corresponding pixel in the CFI with respect to a chosen
  threshold.} \workpackage{Time-series of cloud coverage}{Make plots
  of cloud coverage}.

\vspace{.1cm}

\workpackage{NDVI}{To relate any effects of \elnino{} on human life,
  we will explore whether vegetation is affected. This is done using
  the Normalized Difference Vegetation Index.} \workpackage{Cloud and
  land masks}{Need to mask clouds and sea, so that we only consider
  land.} \workpackage{Time-series of NDVI}{As with cloud coverage,
  produce plots of NDVI to identify correlations with \elnino{}
  events.}

\vspace{.1cm}

\workpackage{SST}{A main indicator of \elnino{} is anomolous increases
  in sea surface temperatures. This is a difficult task, especially
  considering that those temperature changes are of only a few degrees
  celsius.} \workpackage{Improve cloud cover algorithms}{Further
  difficulty arises with having bright clouds on bright seas. Will
  need improved algorithms for determining cloud cover.}
\workpackage{Seasonal means}{As with CFI above, there will be seasonal
  variation that we would like to capture, rather than average out.}

\vspace{.1cm}

\workpackage{Ice caps}{Investigate the effect of \elnino{} on polar
  ice caps. This is a stretch goal and will require supplementary data.}

\vspace{.1cm}

\workpackage{Report writing}{Begin writing report. Ideally we will be
  collecting and assimilating notes while the main investigation is
  underway, so that we are not struggling to remember points of
  interest long after they have been found.}

\vspace{.1cm}

The plan is of course subject to change. There will be work packages
that will overrun, and we may discover new avenues of research to
investigate. In all cases, the plan will be updated accordingly.

\end{document}
