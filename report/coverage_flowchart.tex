% Flowcharting techniques for easy maintenance
% Author: Brent Longborough
\documentclass[x11names]{article}
\usepackage{tikz}
\usetikzlibrary{shapes,arrows,chains}
\begin{document}
% =================================================
% Set up a few colours
\colorlet{lcfree}{Green3}
\colorlet{lcnorm}{Blue3}
\colorlet{lccong}{Red3}
% -------------------------------------------------
% Set up a new layer for the debugging marks, and make sure it is on
% top
\pgfdeclarelayer{marx}
\pgfsetlayers{main,marx}
% A macro for marking coordinates (specific to the coordinate naming
% scheme used here). Swap the following 2 definitions to deactivate
% marks.
\providecommand{\cmark}[2][]{%
  \begin{pgfonlayer}{marx}
    \node [nmark] at (c#2#1) {#2};
  \end{pgfonlayer}{marx}
  } 
\providecommand{\cmark}[2][]{\relax} 
% -------------------------------------------------
% Start the picture
\begin{tikzpicture}[%
    >=triangle 60,              % Nice arrows; your taste may be different
    start chain=going below,    % General flow is top-to-bottom
    node distance=6mm and 60mm, % Global setup of box spacing
    every join/.style={norm},   % Default linetype for connecting boxes
    ]
% ------------------------------------------------- 
% A few box styles 
% <on chain> *and* <on grid> reduce the need for manual relative
% positioning of nodes
\tikzset{
  base/.style={draw, on chain, on grid, align=center, minimum height=4ex},
  proc/.style={base, rectangle, text width=8em},
  test/.style={base, diamond, aspect=2, text width=5em},
  term/.style={proc, rounded corners},
  % coord node style is used for placing corners of connecting lines
  coord/.style={coordinate, on chain, on grid, node distance=6mm and 25mm},
  % nmark node style is used for coordinate debugging marks
  nmark/.style={draw, cyan, circle, font={\sffamily\bfseries}},
  % -------------------------------------------------
  % Connector line styles for different parts of the diagram
  norm/.style={->, draw, lcnorm},
  free/.style={->, draw, lcfree},
  cong/.style={->, draw, lccong},
  it/.style={font={\small\itshape}}
}
% -------------------------------------------------
% Start by placing the nodes
\node [proc, densely dotted, it] (p0) {Calculating multiband monthly cloud coverage};
% Use join to connect a node to the previous one 
\node [proc, join]      {Count number of land pixels $n_l$ in region};
\node [proc, join]      {Load \texttt{threshold} for both bands};
\node [proc, join]      {Load \texttt{image} for both bands};
\node [proc, join]      {\texttt{for} $day$ in $month$};
\node [proc, join]      {\texttt{for} each band};
\node [proc, join] (p1) {Compare $\texttt{mask}(\mathrm{band}_i) = \texttt{image} > \texttt{threshold}$};
\node [proc, join] (p2) {Calculate final \texttt{mask} as $\texttt{mask}(\mathrm{band}_1)$ \texttt{and} $\texttt{mask}(\mathrm{band}_2)$};
\node [proc, join] (p3) {Cloud fraction $c = \texttt{sum(mask)}/n_l$};

% No join for exits from test nodes - connections have more complex
% requirements
% We continue until all the blocks are positioned
% All the other connections come out of tests and need annotating
% First, the straight north-south connections. In each case, we first
% draw a path with a (consistently positioned) annotation node, then
% we draw the arrow itself.

% -------------------------------------------------
\end{tikzpicture}
% =================================================
\end{document}
