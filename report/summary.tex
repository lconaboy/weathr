\section{Summary}

Our objectives with this study were to investigate methods of
forecasting climatological phenomena such as drought and flooding in
regions of Africa where such events may lead to significant impact on
human life. Particularly, we were looking for a connection between
ENSO -- a well studied set of tropical Pacific climate anomalies --
and the trends in cloud cover and NDVI in South Africa and East
Africa.

There exist now many studies with the same idea, each showing various
degrees of ENSO correlation. A study of note is that of
\cite{anyamba2002}, whose study of NDVI anomaly in response to strong
\elnino{} and \nina{} events during the turn of the millenium greatly
informed our own analysis.

We began our analysis by developing a novel method of cloud detection
based on the Otsu method for thresholding. This choice of threshold
was motivated by the hypothesis that any given Earth pixel in the
satellite images we used would present a bimodality in pixel value
over time, with a low pixel value peak corresponding to the ``true''
value for that pixel, and a second peak at larger values corresponding
the the days when cloud was present over that pixel (thus showing a
higher pixel value). The efficacy of this method is certaintly
dependent on the particular region being studied: if a pixel is in a
frequently cloudy location, then it will show the above
bimodality. However for most regions of interest cloudy days are not
as likely as clear days, and so the the distribution of pixel values
will appear to be log-normal rather than bimodal.\footnote{We posit
  that this method would also be ineffective in a similar analysis for
  England, though for the opposite reason: where this report studies
  regions with generally clear skies, England has almost entirely
  cloudy skies.}

Using our thresholds we produced cloud masks. The cloud coverage
calcaulated using these cloud masks was found to correlate well with
published rainfall data, exhibiting the same seasonal modulation. When
comparing to the ONI, we found that cloud fraction and ONI were
generally anticorrelated for South Africa and correlated for East
Africa, although in some cases the signal was weak. This is the
expected signal from the literature. We also found that the anomalies
in cloud fraction followed quite closely the Indian Ocean SSTAs,
especially in East Africa for WTIO anomalies during the mature phase
of \elnino{}. This supports the theory that \elnino{} can impact the
East African climate through warming the Indian Ocean.

Our temporal NDVI analysis led us to similar conclusions as for the
cloud coverage analysis, namely that we found the NDVI and ONI to be
anticorrelated for South Africa and correlated for East Africa. Our
spatial NDVI analysis provided us with some results that were
consistent with the literature, but also others that were quite
different.

The weakness of signals and producing results inconsistent with the
literature, could be effects of both using a small baseline and taking
box averages over areas where regional effects are important (this is
particularly the case with East Africa).

In future work we suggest the employment of other bands in the
production of thresholds. In particular, the use of near-infrared
bands (whose spectral response shows characteristics different to
those of the visual bands) and temperature thresholds could greatly
improve the accuracy of our cloud masks. It also strikes us that cloud
identification is perhaps an ideal task for the burgeoning field of
machine learning: there exists a great quantity of cloud mask data
available that may be utilised for training data. 

Remote sensing has provided great leaps in the understanding of
climate science and the global weather forecasting, though there
remains work to be done in both topics.

%% Local Variables:
%% fill-column: 70
%% TeX-master: "report"
%% End:
