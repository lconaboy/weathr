\section{Summary}

Our objectives with this study were to investigate methods of
forecasting climatological phenomena such as drought and flooding in
regions of Africa where such events may lead to significant impact on
human life. Particularly, we were looking for a connection between
ENSO -- a well studied set of tropical Pacific climate anomalies --
and the trends in cloud cover and NDVI in South Africa and East
Africa.

There exist now many studies with the same idea, each showing various
degrees of ENSO correlation. A study of note is that of
\cite{anyamba2002}, whose study of NDVI anomaly in response to strong
\elnino{} and \nina{} events during the turn of the millenium greatly
informed our own analysis.

We began our analysis by developing a novel method of cloud detection
based on the Otsu method for thresholding. This choice of threshold
was motivated by the hypothesis that any given Earth pixel in the
satellite images we used would present a bimodality in pixel value
over time, with a low pixel value peak corresponding to the ``true''
value for that pixel, and a second peak at larger values corresponding
the the days when cloud was present over that pixel (thus showing a
higher pixel value). The efficacy of this method is certaintly
dependent on the particular region being studied: if a pixel is in a
frequently cloudy location, then it will show the above
bimodality. However for most regions of interest cloudy days are not
as likely as clear days, and so the the distribution of pixel values
will appear to be log-normal rather than bimodal.\footnote{We posit
  that this method would also be ineffective in a similar analysis for
  England, though for the opposite reason: where this report studies
  regions with generally clear skies, England has almost entirely
  cloudy skies.} Using our method for thresholding, we produced cloud
masks of questionable accuracy.

The cloud masks were then used to analyse trends in both cloud
fraction and NDVI.

%% Local Variables:
%% fill-column: 70
%% TeX-master: "report"
%% End:
