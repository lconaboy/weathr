% Move stuff about ENSO events to results?
\section{Discussion}
\label{sec:disc}
\subsection{Thresholding}
\label{sec:disc:thresh}
\subsection{Cloud coverage}
\label{sec:disc:cc}
\subsubsection{Rainfall}
\label{sec:disc:rain}
\subsubsection{SSTA}
\label{sec:disc:ssta}
Here we will discuss how cloud coverage responds to SSTAs, beginning
with South Africa (Figure \ref{fig:cf_t_south}). Between July 2009 --
March 2010 and November 2014 -- May 2016 the ENSO is in its warm phase,
\elnino{}. If there is a response to the positive ONI in 2009/2010,
then Figure \ref{fig:cf_t_south} indicates that it is insignificant in
our data. During the 2014--2016 \elnino{} period we observe negative
cloud fraction anomalies from November 2014 which switch to positive
cloud fraction anomalies around January 2015. In July 2015, the cloud
fraction anomalies peak and turn around again, steadily decreasing
toward a minimum in March 2016, roughly 3 months after the ONI peaked
in December 2015. Important to note is that in the same time period
that the positive cloud fraction anomalies peak, strong positive SWIO
SSTAs $>0.5\degc$ are also present. Perhaps here the dearth of cloud
coverage that we expect to see in South Africa due to an \elnino{}
event is countered by enhanced SSTAs in the SWIO before the strong ONI
again triumphs in suppressing cloud coverage.

In our time range there are technically four \nina{} periods: November
2008 -- March 2009, June 2010 -- May 2011, July 2011 -- March 2012 and
August 2016 -- December 2016. Two of the periods are separated by only
a month where the ONI dipped to $-0.4\degc$. Following the peak of the
2008/2009 \nina{} in January 2009, there was a significant spike in
the cloud fraction in March 2009.  This positive anomaly is present
despite negative SWIO SSTAs. Similarly, following the peak of the
2010/2011 \nina{} in October 2011, there is a significant positive
cloud fraction anomaly in December of the same year. The 2011/2012 and
2016 \nina{} events both precede positive cloud fraction anomalies by
two months or so, however, due to the large uncertainty on the values,
it would be misguided to label these as significant. It is also worth
noting that the ONI magnitude for the 2011/2012 and 2016 \nina s are
relatively small in comparison to the other events in our time period.

We move now to East Africa, shown in Figure
\ref{fig:cf_rf_east}. Immediately before and during the growth of the
2009/2010 \elnino{} we observe negative cloud fractions. Around
October 2009 the cloud fraction anomalies climb rapidly before peaking
in January 2010 at the largest value we observe in our entire dataset
(CF$_{\sigma}=0.88\pm0.59$). This peak in cloud anomalies comes a month
after the peak of the \elnino{} event in December 2009 and coincides
with positive WTIO SSTAs ($>0.25\degc$), present since March 2009
which become large ($>0.5\degc$) in January 2010. Comparing this
response with that of the 2014--2016 \elnino{}, we find that the
negative anomalies present during the growth of \elnino{} are much
more pronou/ nced. The negative cloud anomalies persist from October
2014 to S/eptember 2015 and are accompanied in by a dip in WTIO SSTAs
from December 2014 to February 2015, whereupon the WTIO SSTAs become
positive and remain so until June 2016. The interesting point here is
that the cloud fraction anomalies remain negative for so long despite
both strong ONI and WTIO SSTAs being present. \cite{parhi2016} suggest
a mechanism for this in that troical eastern Africa is affected most
greatly by the mature phase of \elnino{}, through warming of the
neighbouring Indian Ocean (reflected in the positive WTIO SSTAs
between February 2015 and June 2016). The suppression of cloud
coverage evident in the growth phases of both \elnino{} events could be
due to the coincidence of this phase and the `dry' season of East
Africa (shown in Figure \ref{fig:cf_rf_east}), exacerbating the
already present dearth of precipitation. Whatever the mechanism, it is
an interesting result.

Our data show no significant response to the 2008/2009 \nina{} in East
Africa. For the both the 2010/2011 and 2011/2012 \nina events, the
cloud f/raction anomaies and WTIO SSTAs appear to closely follow the
ONI, after a lag. Following the \nina event that begins in June 2010,
negative cloud fraction anomalies begin to appear in October 2010
after transition from the strong positive anomalies of January
2010. The ONI then dips to $-0.4\degc$ in June 2011 and the cloud
fraction anomalies become positive again. This \nina{} event then has
something of a resurgence, the next month dropping back below
$-0.5\degc$, with the minimum occurring in October/November
2011. Cloud fraction anomalies become negative again in November 2011
and they become quite negative, falling to $-0.4$ in February 2012,
until April 2012 whereupon they revert to being positive. Throughout
this, the WTIO SSTAs are closely tracking the cloud fraction
anomalies. This would appear to indicate that if the \nina{} is
affecting the cloud coverage, it is also affecting the SSTAs. For the
2010/2011 \nina{}, the minimum WTIO SSTAs precedes the cloud fraction
anomaly minimum by one month, while for the 2011/2012 \nina{} they
fall on the same month. This could indicate that \nina{} reduces cloud
coverage in East Africa via lowering SSTs in the WTIO.

% concluding remarks

\subsection{Vegetation} 
