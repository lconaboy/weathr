\section{Introduction}

The importance of weather forecasting cannot be understated. We regularly
consult weather forecasts to inform our clothing choices, to help us decide
whether a family outing should be a trip to the beach or to a museum; should we
bring along an umbrella? is it shirt-and-shorts weather?\footnote{Incidentally
  it is never \emph{shirt-and-\textbf{short}-shorts weather.}} These are
legitimate concerns, but they are decidedly \emph{first-world} concerns. In
parts of the world where the amount of rainfall literally means life or death it
is plainly obvious that forewarning of drought is of paramount importance.
Whereas wealthy nations may import food and water in times of need, much of
sub-Saharan Africa relies on the prosperity of local agriculture to meet dietary
needs. Any negative impact on the harvest -- here we concern ourselves only with
climatological causes -- will severely restrict available food sources
\citep{development2006mapping}. Modern Africa has been plagued by socioeconomic
struggles, with frequent civil war, little access to education and medicine, and
highly unstable governments. While much progress has been made in recent
years\footnote{For more information see \url{https://africaindata.org}} the
extra stresses of food shortages and drought may trigger a return to
instability.
% TODO Provide actual citation for above. Is the last line here a
% bit... insensitive?
In 2016 the UN Office for the Coordination of Humanitarian Affairs (UNOCHA)
published a report on the response to \elnino{} in East and Southern Africa
\citep{unocha2016}; by their estimates over 19.5 million people and 10.5
children were affected in East Africa alone. As a result of \elnino{} parts of
East Africa received below average rainfall leading to poor harvests and food
shortages. Similarly, parts of Southern Africa experienced the worst drought for
35 years and leading to huge food insecurity. As one of the worst hit countries,
Kenya alone now has over 1.2 million people in a food security crisis.

Malaria has long been a leading cause of death in Africa, particularly in
children, accounting for approximately 18\% of total deaths in children
\citep{IMHE2016}. \citet{alles1998} states that, at the time of reporting,
malaria transmission intensities---a measure of how effectively malaria is
transmitted in a population---are often two orders of magnitude greater than other
regions where malaria is a significant problem. \cite{loevinsohn1994}
investigated the relationship between climatic warming and malaria incidence
rates in Rwanda, East Africa. The report found that during the particularly warm
period of the late 1980s malaria incidence rates almost doubled, with even
regions previously free of the disease showing an up-take in
incidence.\footnote{Interestingly, the report notes that while there was no
  significant trend in precipitation over the same period, there was heavy rain
  in the years 1987 and 1988 which coincided with a strong ENSO event.}

More recently \cite{craig2004} sought to quantify the association between
various climatic factors---including rainfall and temperature---and malaria
incidence. The report found that \emph{total seasonal cases} of malaria were not
driven by climatological factors; however there was a strong correlation between
interseasonal variability and factors such as the maximum temperatures of the
preceeding season.


% For this reason, it is imperative that we improve our understand of and ability
% to predict this effect in order that we may better prepare and coordinate
% responses to prevent humanitarian disasters.

%% % Would like to say something here referencing data on drought related
%% % deaths in Africa.
%% Indeed forewarning of any impending weather is
%% important for different reasons depending on the local agriculture:
%% forecasting drought allows for stockpiling of water and foods;
%% forecasting of inclement weather allows for farmers to prepare for a
%% greater yield, and locals to prepare for possible flooding.
%% % What other reasons are there for forecasting? This needs rewriting
%% % anyway, as it is a bit ugly in terms of wording. Could be more poetic.
%% Where discussion of the weather is not ``should I wear a raincoat?''
%% but ``will we have enough drinking water?'' weather forecasting should
%% be considered a humanitarian imperative. 
% Is there something to say about global warming here? Probably.

\vspace{1cm}

Early twentieth century observations of atmosphere pressures showed a peculiar
relationship between those measurements in the western tropical Pacific and
those in the southeastern tropical Pacific \citep{holton1989}. Namely, that they
were out of phase --- when one measure was positive, the other was negative.
This was termed the \emph{Southern Oscillation}. Later studies \ref{TODO} would
show that there were accompanying variations in rainfall, sea surface
temperatures, and wind patterns. The combination of these effects would come to
be known collectively as the \elnino{} \emph{Southern Oscillation}, with the
warm phase named \elnino{} and the cold phase \nina{}.


\subsection{Southern Oscillation}
Since the late 19th century, the existence of a large scale `seesaw' in oceanic
surface pressure across the Pacific had been alluded to
\citep{trenberth2000}. The essence of the teleconnection was that when pressure
is high in the Pacific Ocean, pressure tends to be lower in the Indian Ocean
\citep{philander1990}. It was Walker and Bliss who, in the 1930s, characterised
this pattern using measures such as sea level pressure and precipitation, naming
it the Southern Oscillation (SO). However, the interannual pressure fluctations
driving the SO were irregular and there were not enough data for Walker to
determine whether the ocean was involved in the system.

\cite{bjerknes1969} proposed the currently accepted model for atmospheric
circulation driving the SO, calling it the Walker circulation. In this model,
dry air sinks over the cool water of the eastern tropical Pacific. After sinking
it is transported westward along the equator by the trade winds. As it travels
over progressively warmer water the air is warmed and moistened, until it
finally reaches the western tropical Pacific. Here the air is now very warm and
saturated with water, and it rises in prodigious rain clouds. The circulation is
completed with the return flow of air through the upper trophosphere.

\begin{figure*}
  \centering
  \label{fig:slp_corr}
  \includegraphics[width=0.67\textwidth]{figures/slp_corr}
  \caption{Sea level pressure correlations with Southern Oscillation Index, a
    measure of the SO devised by Walker. It is clear to see that the eastern and
    western tropical Pacific ocean are anticorrelated. Figure taken from
    \cite{trenberth2000}.}
\end{figure*}

\subsection{\elnino-Southern Oscillation}
Description of oscillatory ocean-atmospher system. Outline stable and
self-sustained frameworks. Reversal into \nina. Discuss periodicity.

\subsection{Indian Ocean}
Discuss influence on African climate (fairly local). Mention arresting \elnino
2014.

\subsection{Teleconnections}
Describe shifting of circulations as a possible cause for
teleconnections. Evidence for NDVI and rainfall effects (Anyamba etc ...).
%% Local Variables:
%% fill-column: 80
%% End:
