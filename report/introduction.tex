\section{Introduction}

The effects of the \elnino{}--Southern Oscillation (ENSO) are now known to be
significant both locally \emph{and} globally. ENSO has long been known to have
powerful effects local to the tropical Pacific ocean and teleconnected regions:
ocean temperatures and atmospheric pressures vary anomalously high or low,
leading to variation in precipitation, cloud cover, and vegetation cover
\citep{philander1990}. ENSO events have also been shown to affect storm activity
\citep{wang2002}, and are suggested as a contributing factor in cyclogenesis
\citep{sobel2000}.

\cite{trenberth2012} noted how the strong Russian heat wave of 2010, and
numerous other events, were caused by a combination of particularly strong ENSO
and global warming events occurring simultaneously. \cite{bronnimann2007}
reviewed studies relating to the effects of ENSO on European climate and stated
``ENSO is arguably the most important global climate pattern.'' The review
determined that many Europe-focused studies found clear evidence for ENSO
affecting European climate. Similar studies have appeared with a focus on
African climate, with many again suggesting that ENSO has effects on rainfall
\citep{kane2009} and vegetation \citep{anyamba1996, anyamba2001, anyamba2002},
though in all cases the climatological response isn't always present or strong.

\vspace{0.5cm}

Modern Africa has been plagued by socioeconomic struggles, with frequent civil
war, little access to education and medicine, and highly unstable
governments. While much progress has been made in recent years\footnote{For more
  information see \url{https://africaindata.org}} the extra stresses of food
shortages and drought may trigger a return to instability.
% TODO Provide actual citation for above. Is the last line here a
% bit... insensitive?
In 2016 the UN Office for the Coordination of Humanitarian Affairs (UNOCHA)
published a report on the response to \elnino{} in East and Southern Africa
\citep{unocha2016}; by their estimates over 19.5 million people and 10.5
children were affected in East Africa alone. As a result of \elnino{} parts of
East Africa received below average rainfall leading to poor harvests and food
shortages. Similarly, parts of Southern Africa experienced the worst drought for
35 years, leading to huge food insecurity. As one of the worst hit countries,
Kenya alone now has over 1.2 million people in a food security crisis.

Malaria has long been a leading cause of death in Africa, particularly in
children, accounting for approximately 18\% of total deaths in children
\citep{IMHE2016}. \citet{Alles1998} states that, at the time of reporting,
malaria transmission intensities---a measure of how effectively malaria is
transmitted in a population---are often two orders of magnitude greater than other
regions where malaria is a significant problem. \citet{loevinsohn1994}
investigated the relationship between climatic warming and malaria incidence
rates in Rwanda, East Africa. The report found that during the particularly warm
period of the late 1980s, malaria incidence rates almost doubled, with even
regions previously free of the disease showing an up-take in
incidence\footnote{Interestingly, the report notes that while there was no
  significant trend in precipitation over the same period, there was heavy rain
  in the years 1987 and 1988 which coincided with a strong ENSO event.}.

More recently, \cite{craig2004} sought to quantify the association between
various climatic factors---including rainfall and temperature---and malaria
incidence. The report found that \emph{total seasonal cases} of malaria were not
driven by climatological factors; however there was a strong correlation between
interseasonal variability and factors such as the maximum temperatures of the
preceding season.

\vspace{0.5cm}

The importance of weather forecasting cannot be understated. We regularly
consult weather forecasts to inform our clothing choices, to help us decide
whether a family outing should be a trip to the beach or to a museum; should we
bring along an umbrella? is it shirt-and-shorts weather?\footnote{Incidentally
  it is never \emph{shirt-and-\textbf{short}-shorts weather.}} These are
legitimate concerns, but they are decidedly \emph{first-world} concerns. In
parts of the world where the amount of rainfall literally means life or death it
is plainly obvious that forewarning of drought is of paramount importance.
Whereas wealthy nations may import food and water in times of need, much of
sub-Saharan Africa relies on the prosperity of local agriculture to meet dietary
needs. Any negative impact on the harvest -- here we concern ourselves only with
climatological causes -- will severely restrict available food sources
\citep{development2006mapping}.

A deep understanding of the interplay between the effects of ENSO upon African
climate and in turn the knock-on humanitarian consequences should be considered
a scientific imperative. By showing that ENSO correlates with any African
climate phenomena, we will have provided a useful method for forecasting, and
thus also for preparing for any necessary humanitarian relief. It is with this
goal that the following study is motivated.

%% Local Variables:
%% fill-column: 80
%% TeX-master: "report"
%% End:
