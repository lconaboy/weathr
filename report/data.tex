\section{Data}
% are all the acronyms too long?
% TODO diurnal (?)
% TODO regions
% TODO figure for regions
% TODO SST anomalies
% TODO rainfall data
\begin{table}
  \centering
  \resizebox{0.85\linewidth}{!}{%
  \begin{tabular}{ l c c c }
    \toprule
    Channel & $\lambda_{\mathrm{cen}}$ ($\mu$m) & $\lambda_{\mathrm{min}}$ ($\mu$m) & $\lambda_{\mathrm{max}}$ ($\mu$m) \\
    \midrule
    VIS0.6  & 0.64                    & 0.56                    & 0.71                    \\
    VIS0.8  & 0.81                    & 0.74                    & 0.88                    \\
    NIR1.6  & 1.64                    & 1.50                    & 1.78                    \\
    \bottomrule
  \end{tabular}}
  \caption{Spectral characteristics of the three SEVIRI channels used
    in our work. Shown are the central, minimum and maximum
    wavelengths for the three spectral bands that we use.}
  \label{tab:seviri}
\end{table}

This work mainly comprises of analysing data products that we have
produced using raw data from the European Organisation for the
Exploitation of Meteorological Satellites
(EUMETSAT) \footnote{\url{https://www.eumetsat.int}}. In particular,
we have utilised the Meteosat Second Generation (MSG) series of
meterological satellites, which provide continous observations of the
full disk of the Earth via the Spinning Enhanced Visible and Infrared
Imager (SEVIRI) \citep{schmetz2002}. MSG makes observations every 15
minutes in 12 spectral bands, returning images $3712 \times 3712$ pixels in
size. Not all of the 12 available bands are useful for our
investigation -- the spectral responses of the three bands that we use
are detailed in Table \ref{tab:seviri}.

\begin{table}
  \centering
  \resizebox{\linewidth}{!}{%
  \begin{tabular}{ l c c c c }
    \toprule
    \multirow{2}{*}{Region} &
    \multicolumn{2}{c}{Latitude (deg)} &
    \multicolumn{2}{c}{Longitude (deg)} \\
                 & {North} & {South} & {East} & {West}    \\
    \midrule
    South Africa & $-21.501$ & $-35.659$ & 33.193 & 14.728  \\
    East Africa  & 10.085    & 1.615     & 46.071 & 34.518  \\
    \bottomrule
  \end{tabular}}
  \caption{The coordinates that define our regions of interest. North
    (East) and South (West) correspond to the most northern (eastern)
    and southern (western) lines of latitude (longitude),
    respectively. All values in decimal degrees.}
\end{table}


